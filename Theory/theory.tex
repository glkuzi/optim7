% !TeX spellcheck = ru_RU
\documentclass[12pt]{report}
\usepackage[a4paper, top=2.5cm]{geometry}
%\usepackage{fancyhdr}
\usepackage[T2A]{fontenc}
\usepackage[utf8]{inputenc}                       % if you are not using xelatex ou lualatex, replace by the encoding you are using
%\usepackage{sansmathaccent}
%\pdfmapfile{+sansmathaccent.map}
\usepackage{amsmath}
%\usepackage{amsfonts}
\usepackage[english,russian]{babel}
\usepackage{graphicx}
\graphicspath{{pictures/}}
\DeclareGraphicsExtensions{.pdf,.png,.jpg}
\usepackage{caption}
\usepackage{textpos}
\usepackage{paracol}
\usepackage{hyperref}
\hypersetup{
	colorlinks=true,
	linkcolor=blue,
	filecolor=magenta,      
	urlcolor=cyan,
}

\newenvironment{myparacol}[2][]{%
	\begin{paracol}{#2}[#1]\setlength{\parindent}{0pt}}{%
	\end{paracol}}
\begin{document}

%\begin{comment}

\begin{titlepage}
	%\setcounter{page}{1}
	%\thispagestyle{empty}
	\begin{center}
		Федеральное государственное автономное образовательное\\ учреждение
		высшего образования\\
		"Московский физико-технический институт\\ (национальный исследовательский университет)"\\
		Физтех-школа прикладной математики и информатики\\
		Кафедра системных исследований\\
	\end{center}
	\textbf{Направление подготовки:} 03.04.01 Прикладные математика и физика\\
	\textbf{Направленность (профиль) подготовки:} Методы и технологии искусственного интеллекта
	\begin{center}
		\vspace{4cm}
		\textbf{Проект по методам оптимизации №7}\\
		\vspace{4cm}
	\end{center}
	\begin{myparacol}{2}
		\switchcolumn
		\textbf{Студенты:}\\
		Кузьмин Глеб Юрьевич\\
		Шарафутдинов Якуб Насырьянович\\
	\end{myparacol}
	\begin{center}
		\vspace{6cm}
		\textbf{г. Долгопрудный\\
			2020}
	\end{center}
\end{titlepage}

%\end{comment}
\paragraph{}Рассматривается функция Нестерова-Скокова:
\begin{equation}\label{Nesterov-Skokov}
f(x)=\frac{1}{4}(x_{1}-1)^{2}+\sum_{i=1}^{n-1}(x_{i+1}-2x_{i}^{2}+1)^{2}.
\end{equation} 
Экстремум функция достигает в точке глобального минимума $x_{*}=(1,1,...,1)$, $f(x_{*})=0$.

Требуется доказать, что функция Нестерова-Скокова удовлетворяет условию Поляка-Лоясиевича:
\begin{equation}\label{condition}
f(x)-f(x_{*})\leq\frac{1}{2\mu}||\nabla f(x)||_{2}^{2}.
\end{equation}
Используя переход от функции к системе уравнений, использованный в \href{https://arxiv.org/pdf/1711.00394.pdf}{пособии} (стр. 27), получим матрицу Якоби для системы уравнений $g$. В нашем случае функция $f(x)$ соответствует системе уравнений $g$ следующего вида: $x_{1}=1$, $x_{2}=2x_{1}^{2}-1$, $x_{3}=2x_{2}^{2}-1$, ..., $x_{n}=2x_{n-1}^{2}-1$. Для оценки числа обусловленности воспользуемся условием из пособия:
\begin{equation}\label{condition2}
\lambda_{min}(\partial g(x)/\partial x\cdot [\partial g(x)/\partial x]^{T})\geq\mu
\end{equation}
и верхней оценкой для $L$: $L\geq tr(\partial g(x)/\partial x\cdot [\partial g(x)/\partial x]^{T})$. Обозначим матрицу $\partial g(x)/\partial x\cdot [\partial g(x)/\partial x]^{T}$ как $G$, тогда:
\begin{equation}\label{matrix}
G= \begin{pmatrix}
	1 & 0 & 0 & ... & 0\\
	4x_{1} & (4x_{1})^{2} + 1 & 0 & ... & 0\\
	16(2x_{1}^{3}-x_{1}) & 4x_{1}(16(2x_{1}^{3}-x_{1})) + 4x_{2} & (16(2x_{1}^{3}-x_{1}))^{2} + (4x_{2})^{2} + 1 & ... & 0\\
	... & ... & ... & ... & ...\\
	\frac{\partial x_{n}}{\partial x_{1}} & ... & ... & ... & (\frac{\partial x_{n}}{\partial x_{1}})^{2} + ... + 1
   \end{pmatrix}.
\end{equation}
Таким образом получим $\mu=1$ и $L=((4x_{1})^{2} + 1)\cdot((16(2x_{1}^{3}-x_{1}))^{2} + (4x_{2})^{2} + 1)\cdots((\frac{\partial x_{n}}{\partial x_{1}})^{2} + ... + 1)$. В итоге мы получили что функция $f(x)$ удовлетворяет условию Поляка-Лоясиевича, но с очень большим числом обусловленности $\frac{L}{\mu}$.


\end{document}